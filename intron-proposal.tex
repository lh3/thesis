\newif\ifPDF
\ifx\pdfoutput\undefined\PDFfalse
\else\ifnum\pdfoutput > 0\PDFtrue
	\else\PDFfalse
	\fi
\fi

\ifPDF
	\documentclass[pdftex,11pt,openany,twoside]{book}
	\RequirePackage[hyperindex,colorlinks,plainpages=false]{hyperref}
	\hypersetup{pdfauthor={Heng Li},linkcolor=blue,citecolor=blue,urlcolor=blue}
	\usepackage{graphicx}
	\DeclareGraphicsRule{*}{mps}{*}{}
\else
	\documentclass[11pt,openany,twoside]{book}
	\usepackage{graphicx}
	\newcommand{\href}[2]{\textcolor{MYBLUE}{#1}}
\fi


\usepackage[%paperwidth=18.4cm, paperheight=26cm,
body={14.6true cm,22true cm},
twosideshift=0 pt,
%headheight=1.0true cm
]{geometry}

\usepackage{caption2}
\usepackage{fancyhdr}
\pagestyle{fancy}
% with this we ensure that the chapter and section
% headings are in lowercase.
\renewcommand{\chaptermark}[1]{\markboth{\chaptername \ \thechapter. \ #1}{}}
%\renewcommand{\sectionmark}[1]{\markright{\thesection\ #1}}
\fancyhf{} % delete current setting for header and footer
\fancyhead[LE,RO]{\bfseries\thepage}
\fancyhead[LO]{\bfseries\leftmark}
\fancyhead[RE]{\bfseries Constructing TreeFam Database}
\renewcommand{\headrulewidth}{0.5pt}
\renewcommand{\footrulewidth}{0pt}
\addtolength{\headheight}{0.5pt} % make space for the rule
\addtolength{\headwidth}{10pt} % make space for the rule
\fancypagestyle{plain}{%
\fancyhead{} % get rid of headers on plain pages
\renewcommand{\headrulewidth}{0pt} % and the line
}

\sloppy

\usepackage{amsmath,amssymb,amsfonts}
\usepackage[usenames,dvips]{color}
\usepackage{makeidx}

\usepackage{flafter}
\usepackage[below]{placeins}
\usepackage{floatflt}

\definecolor{MYRED}{rgb}{1,0,0}
\definecolor{MYBLUE}{rgb}{0,0,1}
%\newcommand{\lhcomm}[1]{\textcolor{MYRED}{#1}}
\newcommand{\lhcomm}[1]{#1}

\addtolength{\headsep}{-0.1cm}
\addtolength{\footskip}{-0.1cm} 

\renewcommand{\textfraction}{0.15}
\renewcommand{\topfraction}{0.85}
\renewcommand{\bottomfraction}{0.65}
\renewcommand{\floatpagefraction}{0.60}

\renewcommand{\captionfont}{\fontsize{10pt}{12pt}\selectfont}

\addtolength{\oddsidemargin}{1.2cm}
\addtolength{\evensidemargin}{-1.2cm}

%\addtolength{\textwidth}{3cm}
%\addtolength{\hoffset}{-1.5cm}
%\addtolength{\textheight}{4cm}
%\addtolength{\voffset}{-2cm}

\makeindex

\usepackage{bioinformatics}
\bibliographystyle{bioinformatics}

\title{Working Proposal for Intron Evolution}
\author{Heng Li}

\begin{document}
\maketitle

\section{Origin of Introns}

\subsection{Background}
``Intron early'' hypothesis states that
spliceosomal introns were present in the ancestor of prokaryotes and eukaryotes,
while ``intron late'' hypothesis argues introns were originated after the
speciation of the two kingdoms. Although several large-scale
analyses~\cite{roy05,nguyen05}, which are based on 684 one-to-one orthologous
gene families constructed by~\cite{rogozin03}, have been done to study which one is the fact,
the answser remains unclear~\cite{rogozin05}.

In the lack of a prokaryotic outgroup in TreeFam, our project cannot resolve
this long-lasting debate, either. But with more species and gene families involved,
we will get a refined picture of intron losses and gains in smaller lineages.
In addtion, inclusion of paralogs also makes it possible to study intron
evolution between paralogs.

\subsection{Proposed Methods}
Rebuild TreeFam multialignment only containing sequenced species to get better alignment. Map
well aligned introns to each internal node of the trees with missing values allowed. Count
intron gain/loss between two successive speciation nodes and do statistics. Intron loss/gain
between paralogs can be achieved by examining duplication nodes. Also refer to
Babenko {\it et al.}~\cite{babenko04}.

\section{Mechanisms of Intron Losses and Gains}

\subsection{Background}
The mechanisms of intron
losses are relatively well studied. Two factors, reverse transcription
and genomic deletion, are thought to cause losses.
In comparison, the mechanisms of intron gains are much more complex.
They can be explained by intron transposition, transposon insertion,
tandem genomic duplication, intron transfer between paralogs and self-splicing group II intron~\cite{roy06,coghlan04}.
Concrete evidence has been found for all mechanisms,
but it is still unclear how often they play a role in evolution.

\subsection{Proposed Methods}
\subsubsection{Locating genomic deletions}
Genomic deletions can be observed from alignment if gaps exist around a splicing boundary.
But sometimes they can be mixed with alignment errors.
\subsubsection{Studying intron gains}
Align new introns with other conserved introns to detect intron transposition.
(See Coghlan {\it et al.}~\cite{coghlan04} for more.)

\section{Patterns of Intron/Gene Size Evolution}

\subsection{Background}
It is known that genes with smaller genomic size
tend to have higher expressional levels~\cite{castilloDavis02,urrutia03}.
Eisenberg {\it et al.}~\cite{eisenberg03} also pointed out housekeeping genes, which tend to highly expressed, have shorter introns.
Both works attribute this correlation to the ``selection for economy''.
They regard the current patterns of intron sizes as a result of the balance between the selection
of economical transcription and the overall trend of intron expansion due to
transposons insertions. Housekeeping genes are shorter because they are frequently
used and thus subject to higher selective pressures. Recently, Vinogradov~\cite{vinogradov06}
challanged this notion. By examing the expressional level and percentage of conserved regions,
he argued that intron length is determined by the gene functions. This means that intron/gene size
itself is of biological meanings. In other words, intron/gene size is selected for functions
instead of for economy.

\subsection{Proposed Method}
To our preliminary study in BGI, genomic sizes of orthologous mammalian genes are more conservative than sizes
of orthologous introns. If this is the fact (not caused by bugs or artifacts), it confirms that gene size itself, rather than
intron size, is of more importance. This result also supports that gene size is selected for
functions; otherwise we are expected to observe longer introns are less correlated.
We will re-produce the results using TreeFam data and model this process if possible.
By studying paralogs and integrating expressional data, we can make the conclusion more solid.

\bibliography{ref-intron}
\end{document}
